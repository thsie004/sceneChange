\documentclass{beamer}

\begin{document}

	\begin{frame}
	\frametitle{Fourier Analysis}
		\begin{itemize}
			\item Fourier analysis in a nutshell: How do we decompose a general function into a sum of simple trigonometric functions (like sine and cosine)?
			\item More technically, we want to express a function of time (signal space) in terms of the amplitude of the frequencies that make it up (frequency space).
		\end{itemize}
	\end{frame}
	
	\begin{frame}
	\frametitle{How can we apply Fourier Analysis to an image?}
		\begin{itemize}
			\item While an image is not a function, it is a sampling of signals (pixels). Therefore, we can take the Discrete Fourier Transform (DFT) of an image.
			\item In return, we get an image of the same size, but with each pixel replaced by the amplitude of a particular frequency.
			\begin{figure}
				\includegraphics[scale = 0.5]{/Home/sarah/Downloads/cosines1}
			\end{figure}
			\begin{figure}
				\includegraphics[scale = 0.5]{/Home/sarah/Downloads/brks_blks}
			\end{figure}
		\end{itemize}
	\end{frame}

\end{document}
	
			
	