\documentclass{beamer}
\usepackage{amsmath,amssymb,amsthm,amsfonts,commath,tikz}
\usetheme{Warsaw}
\usecolortheme{rose}
\usetikzlibrary{arrows,positioning,fit,matrix,shapes.geometric,external}



\begin{document}


	% SLIDE 1
	% Title
	\section{UCR Undergraduate Research Project} 
	\title{Distribution of the Gaussian primes}
	\author{{\it Melina Fuentes} and {\it Michel Manrique}\hspace{8ex} \\
	\footnotesize Advisors: {\it John Dusel} and {\it Jason Erbele} }
	\institute{University of California, Riverside}
	\maketitle

	\setbeamertemplate{itemize items}[default]
	\setbeamertemplate{enumerate items}[default]

	% SLIDE 2
	% Modular Arithmetic
	\begin{frame}
		\frametitle{Modular Arithmetic}
		\pause	
		\begin{itemize}
			\item 	Language of congruences developed by Karl Friedrich Gauss (1777-1855) 
				at beginning of 19\textsuperscript{th} century.
			\item   First notation that made working with divisibility relationships easier, and less awkward, 
				which in turn, helped accelerate number theory.
		\end{itemize}
		\pause
		\begin{definition}
			Let m be a positive integer. If a and b are integers, we say that 
			a is \textit{congurent to b modulo m} if $m \mid (a-b)$.
			\\ \hspace{4ex} That is, $a \equiv b \pmod m$  if $m \mid (a-b)$.
			\\ \hspace{4ex} However, $a \not\equiv b \pmod m$ if  $m \nmid (a-b)$. [R]
		\end{definition}
		\pause
		\begin{example}
			$11 \equiv 3 \pmod 4$ since $4 \mid (11-3) = 8$.
			\\ $3 \equiv -6 \pmod 9$ since $9 \mid (3-(-6))=9$.
		\end{example}
	\end{frame}

	% SLIDE 3
	% Acknowledgements
	\begin{frame}
		\frametitle{Acknowledgements}
		\pause
		\hspace{4ex} 
		We would like to thank John Dusel, Jason Erbele, Dr. Kevin Costello, Dr. Vyjayanthi Chari, the UCR math club, 
		and the UCR Mathematics department for allowing us the opportunitiy to be apart of this undergraduate mathematics
		research project. We are sincerely grateful for the chance to learn and discover new topics that are not standard to the 
		undergraduate mathematics curriculum. Thank you all!
	\end{frame}

	% SLIDE 4
	% Bibliography
	\begin{frame}
		\frametitle{Bibliography}
		 \begin{thebibliography}{}
			\setbeamertemplate{bibliography item}[text]
			\bibitem{EW} Graham Everest and Thomas Ward, {\em An Introduction to Number Theory},  Springer-Verlag,
	 		Berlin, 2005.
			\bibitem{GG}  Andrew Granville and Greg Martin, {\em Prime Number Races}, The Mathematical Association of 
			America, Monthly 113, January 2006.
			\bibitem{R}     Kenneth H. Rosen, {\em Elementary Number Theory and Its Applications}, Pearson, 2005.
		\end{thebibliography}
	\end{frame}


\end{document}